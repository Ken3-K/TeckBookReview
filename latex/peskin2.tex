\documentclass{ltjsarticle}
\usepackage{amsmath}
\usepackage{amssymb}
\usepackage{bm}

\begin{document}

• p.14 $U(t)$ の計算

時間並進の演算子は $ke^{-iHt}\mathcal{K}$ なので、 $x_{0}$ から $x$ に遷移する確率振幅 $U(t)$ は

\begin{align*}
U(t)&=\langle x|e^{-iHt}|x_{0}\rangle \\
&=\langle x|e^{-i\frac{p^{2}}{2m}}|x_{0}\rangle \\
&=\int\frac{d^{3}p}{(2\pi)^{3}}\langle x|e^{-i\frac{p^{2}}{2m}}|p\rangle\langle p|x_{0}\rangle \quad \text{完全性関係を利用} \\
&=\int\frac{d^{3}p}{(2\pi)^{3}}e^{-i\frac{p^{2}}{2m}}\langle x|p\rangle\langle p|x_{0}\rangle \\
&=\int\frac{d^{3}p}{(2\pi)^{3}}e^{-i\frac{p^{2}}{2m}}e^{ip\cdot(x-x_{0})} \quad (\langle x|p\rangle=e^{ix\cdot p}) \\
&=\int\frac{d^{3}p}{(2\pi)^{3}}e^{-i\frac{t}{2m}(p+\frac{m}{t}(x-x_{0}))^{2}}e^{i\frac{m}{2t}(x-x_{0})^{2}} \\
&=\frac{1}{(2\pi)^{3}}(\frac{2m\pi}{it})^{\frac{3}{2}}e^{i\frac{m}{2t}(x-x_{0})^{2}} \quad \text{ガウス積分} \\
&=\left(\frac{m}{2\pi it}\right)^{\frac{3}{2}}e^{i\frac{m}{2t}(x-x_{0})^{2}}
\end{align*}

4行目では運動量の固有値を取るので、指数関数はブラケットの外に出すことができる。

• p.14二つ目の式の計算

\begin{align*}
\int d^{3}pe^{-it\sqrt{p^{2}+m^{2}}}e^{ip\cdot(x-x_{0})} 
&=\int_{0}^{2\pi}d\varphi\int_{0}^{\infty}p^{2}dp\int_{-1}^{1}d\mu e^{-it\sqrt{p^{2}+m^{2}}}e^{ip|x-x_{0}|\mu} \\
&=2\pi\int_{0}^{\infty}p^{2}dp\frac{1}{ip|x-x_{0}|}e^{-it\sqrt{p^{2}+m^{2}}}(e^{ip|x-x_{0}|}-e^{-ip|x-x_{0}|}) \\
&=4\pi\int_{0}^{\infty}dpp~\sin(p|x-x_{0}|)e^{-it\sqrt{p^{2}+m^{2}}}
\end{align*}

\end{document}